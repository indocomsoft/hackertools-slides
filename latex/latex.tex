\documentclass[12pt]{beamer}

\usetheme[sectionpage=none, subsectionpage=progressbar, progressbar=foot, numbering=fraction]{metropolis}

\usepackage{tabularx}

\makeatletter
\setlength{\metropolis@frametitle@padding}{1.6ex}% <- default 2.2 ex

\setbeamertemplate{footline}{%
  \begin{beamercolorbox}[wd=\textwidth, sep=1.5ex]{footline}% <- default 3ex
    \usebeamerfont{page number in head/foot}%
    \usebeamertemplate*{frame footer}
    \hfill%
    \usebeamertemplate*{frame numbering}
  \end{beamercolorbox}%
}
\makeatother

\AtBeginSubsection
{
  \begin{frame}{Where are we?}
    \tableofcontents[sectionstyle=show/shaded, subsectionstyle=show/shaded/hide]
  \end{frame}
}

\makeatletter
\setbeamertemplate{headline}{
  \begin{beamercolorbox}{upper separation line head}
  \end{beamercolorbox}
  \begin{beamercolorbox}{section in head/foot}
    \vskip2pt\insertsectionnavigationhorizontal{\paperwidth}{}{}\vskip2pt
  \end{beamercolorbox}
  \begin{beamercolorbox}{lower separation line head}
  \end{beamercolorbox}
}
\makeatother
\setbeamercolor{section in head/foot}{fg=normal text.bg, bg=structure.fg}

\setbeamertemplate{itemize items}[square]


\usepackage{menukeys}
\usepackage{minted}
\setminted[latex]{fontsize=\small, tabsize=2, breaklines}

\title{Hacker Tools: \LaTeX{}}
\author{Julius Putra Tanu Setiaji}
\date{3 November 2020 \\ Slides at \url{https://is.gd/ht_latex_2020_slides}}

\begin{document}

\frame[plain]{\titlepage}

\section{Introduction}
\subsection{}

\begin{frame}{NUS Hackers}

  \begin{center}
    \includegraphics[width=0.5\linewidth]{../NUSHackers}

    \url{http://nushackers.org}
  \end{center}

  \begin{center}
    \textbf{hacker}school

    Friday \textbf{Hacks}

    \textbf{Hack} \& Roll

    \textbf{Hacker} Tools
  \end{center}

\end{frame}

\begin{frame}{About Me}
  Hi! I'm Julius. My GitHub is \url{https://github.com/indocomsoft}

  A Year 4 Computer Science Undergraduate who loves hacking and building systems.

  I also enjoy Space Exploration, Music Theory and History.

    {\tiny (my favourite games are KSP and EU4 hit me up if you play those too)}
\end{frame}

\begin{frame}{Required Software}
  These are preferable, but otherwise you should be to follow along using Overleaf\footnote{\url{https://www.overleaf.com/?r=5e9ffb40&rm=d&rs=b}}
  \begin{itemize}
    \item A \TeX{} distribution (instructions in our publicity channels)
    \item TeXstudio
  \end{itemize}
\end{frame}

\begin{frame}{What is \LaTeX{}?}
  \begin{itemize}
    \item A markup language for document preparation\footnote{Just like HTML (Hyper-Text Markup Language) is a markup language}
    \item Uses plain text\footnote{thus versionable using a VCS like \texttt{git}} in contrast to most WYSIWYG editors
    \item Started as a writing tool for mathematicians and computer scientists.
    \item Built on top of \TeX{} by Leslie Lamport\footnote{Winner of the Turing Award in 2013 for his work in distributed and concurrent systems} in 1983
  \end{itemize}
\end{frame}

\begin{frame}{What is \TeX{}?}
  \begin{itemize}
    \item A typesetting system designed and mostly written by Donald Knuth\footnote{Winner of the Turing Award in 1974 for analysis of algorithms and the design of programming languages} in 1978
    \item Because Knuth was disappointed with the typesetting of the 2nd edition of TAOCP.
    \item 2 Goals:
          \begin{itemize}
            \item Allow anybody to produce high-quality books with minimal effort
            \item Provide a system that would give exactly the same results on all computers, at any point in time
          \end{itemize}
  \end{itemize}
\end{frame}

\begin{frame}{Trivia}
  Version number of \TeX{} approaches $\pi$:

  $3.0 \rightarrow 3.1 \rightarrow 3.14 \rightarrow 3.141 \rightarrow ... \rightarrow 3.14159265$ (current)

  Version number of Metafont\footnote{Companion to \TeX{} written by Knuth, used to describe fonts using geometrical equations} approaches $e$:

  $2.0 \rightarrow 2.7 \rightarrow 2.71 \rightarrow ... \rightarrow 2.7182818$ (current)
\end{frame}

\begin{frame}{What can I use \LaTeX{} for?}
  \begin{itemize}
    \item Reports
    \item Books
    \item Presentation\footnote{This presentation is written in \LaTeX{} using Beamer! \url{https://github.com/indocomsoft/hackertools-slides/blob/master/latex/latex.tex}}
    \item And so much more!
  \end{itemize}
\end{frame}

\section{Syntax}
\begin{frame}[fragile]{Basic \LaTeX{} Syntax}
  \begin{itemize}
    \item A \LaTeX{} document consists of commands and environments\footnote{HTML terms: tags = commands, tags with children = environments}
    \item The command syntax: \\
          \mintinline{latex}{\command[option1,option2,...]{arg1}{arg2}...}
    \item The environment syntax:
          \begin{minted}{latex}
\begin{environment}
  % Some children content
\end{environment}
          \end{minted}
    \item Comments are whatever comes after \mintinline{latex}{%}
  \end{itemize}
\end{frame}

\begin{frame}[fragile]{Basic \LaTeX{} Document}
  We will explain the commands and environment used here later on.
  \begin{minted}{latex}
\documentclass{article}

\begin{document}
Hello world!
\end{document}
  \end{minted}
\end{frame}

\begin{frame}{Spaces}
  \begin{itemize}
    \item All whitespace characters are treated as space.
    \item Several consecutive spaces are treated as one space.
    \item Leading/trailing spaces are ignored.
    \item A single line break is treated as a space.
    \item Two or more line breaks define the end of a paragraph.
  \end{itemize}
\end{frame}

\begin{frame}[fragile]{Let's try out spaces}
  \begin{minted}{latex}
\begin{document}
It does not matter whether you
enter one or several             spaces
after a word.

An empty line starts a new
paragraph.
\end{document}
  \end{minted}
\end{frame}

\begin{frame}[fragile]{Reserved Characters}
  Reserved characters either have a special meaning or are unavailable in all the fonts \footnote{This might feel weird, but remember that \TeX{} and \LaTeX{} are such old systems from the 1970s and 1980s}.

  \begin{minted}{text}
# $ % ^ & _ { } ~ \
  \end{minted}

  Instead, use

  \begin{minted}{latex}
\# \$ \% \^{} \& \_ \{ \} \~{} \textbackslash
  \end{minted}

  Note the empty argument to caret and tilde, because otherwise they are used to create diacritics.

  We use \mintinline{text}{\textbackslash} because \mintinline{text}{\\} is line breaking.
\end{frame}

\begin{frame}[fragile]{Other tricky characters}
  \begin{itemize}
    \item Larger than and smaller than symbols usually do not get rendered correctly.
    \item Instead, use \mintinline{latex}{\textless} and \mintinline{latex}{\textgreater}
    \item In some circumstances, square brackets are reserved (for options)
    \item Thus, \mintinline{latex}{\command [text]} fails, instead do \mintinline{latex}{\command{} [text]}
  \end{itemize}
\end{frame}

\begin{frame}{Packages}
  \begin{itemize}
    \item Just like other programming languages, \LaTeX{} has packages as well
    \item \LaTeX{} also has its own package manager, called \texttt{CTAN}
    \item Use the command \mintinline{latex}{\usepackage{packagename}} to ``import'' and use a package.
    \item We will go through some useful packages in the upcoming subsections.
  \end{itemize}
\end{frame}

\section{Commands and Environments}
\subsection{}
\begin{frame}[fragile]{Back to Our Example}
  \begin{minted}{latex}
\documentclass{article}

\begin{document}
Hello world!
\end{document}
  \end{minted}
\end{frame}

\subsection{Document Class}
\begin{frame}[fragile]{Document Class}
  \begin{minted}{latex}
\documentclass{article}
  \end{minted}
  \begin{itemize}
    \item Use the \texttt{article} document class.
    \item Document class file defines the formatting standard to follow, which in this case is the generic article format.
    \item Other document classes, e.g. \texttt{acmart} for ACM\footnote{Association for Computing Machinery} publications, \texttt{beamer} for presentations\footnote{Like this presentation!}
  \end{itemize}
\end{frame}

\begin{frame}{Document Class options}
  \begin{itemize}
    \item \texttt{10pt}, \texttt{11pt}, \texttt{12pt} -- size of main font (default: 10pt)
    \item \texttt{a4paper}, \texttt{letterpaper}, ... - size of paper
    \item \texttt{landscape} -- Landscape mode layout
    \item \texttt{titlepage}, \texttt{notitlepage} -- whether a new page should be started after the document title
  \end{itemize}

  Find out more at \url{https://en.wikibooks.org/wiki/LaTeX/Document_Structure\#Document_classes}
\end{frame}

\subsection{Document environment}
\begin{frame}[fragile]{Document Environment}
  \begin{minted}{latex}
\begin{document}
  \end{minted}
  \begin{itemize}
    \item The beginning of the \texttt{document} environment.
    \item Tells \LaTeX{} that the content of document starts here.
    \item Anything before this line is called \textbf{the preamble}
  \end{itemize}
  \begin{minted}{latex}
\end{document}
  \end{minted}
  \begin{itemize}
    \item The end of the \texttt{document} environment
    \item Tells \LaTeX{} that the document is complete.
    \item Anything after this line is ignored.
  \end{itemize}
\end{frame}

\begin{frame}[fragile]{Top Matter}
  Top Matter: information about the document itself
  \begin{itemize}
    \item Provide information using the \texttt{title}, \texttt{author}, \texttt{date}
    \item Typeset the title using \texttt{maketitle}
  \end{itemize}
  \begin{minted}{latex}
\documentclass{article}

\title{How to Basic: \LaTeX{}}
\author{Julius Putra Tanu Setiaji}
\date{3 November 2020}

\begin{document}
\maketitle
\end{document}
  \end{minted}
\end{frame}

\begin{frame}[fragile]{Sectioning Commands}
  \begin{minted}{latex}
\section{Some Section Title}
\subsection{Some Subsection Title}
\subsubsection{Some Subsubsection Title}
  \end{minted}
  To get an unnumbered sections, add an asterisk to the end of the command name, e.g. \mintinline{latex}{\section*{Look Ma, no numbers!}}

  Typeset a table of contents using \mintinline{latex}{\tableofcontents}

  Note: unnumbered section will not be included in the TOC unless explicitly included:
  \begin{minted}{latex}
\addcontentsline{toc}{subsection}{Look Ma, no numbers!}
  \end{minted}
\end{frame}

\subsection{Fonts}
\begin{frame}[fragile]{Emphasising text}
  \begin{itemize}
    \item Use the \mintinline{latex}{\emph{text}} command
    \item Typically done by italicising the text.
    \item Note that the command is dynamic: emphasising a word in an already emphasised sentence will revert the word to upright font.
  \end{itemize}
\end{frame}

\begin{frame}[fragile]{Font styles}
  \begin{minted}{latex}
\textnormal{document font family}
\emph{Emphasised text}
\texttt{teletype font family (monospaced)}
\textbf{bold fontface}
\textsc{Small Capitals}
\uppercase{uppercase}
  \end{minted}
\end{frame}

\begin{frame}[fragile]{Font size}
  Changes the size in scope
  \begin{minted}{latex}
{\tiny test}
{\scriptsize test}
{\footnotesize test}
{\small test}
{\normalsize test}
{\large test}
{\Large test}
{\LARGE test}
{\huge test}
{\Huge test}
  \end{minted}
\end{frame}

\subsection{Text and Paragraph Formatting}
\begin{frame}[fragile]{Non-breaking Space}
  Use tilde (\mintinline{latex}{~}) to tell \LaTeX{} not to change space into line break.
\end{frame}

\begin{frame}[fragile]{Line spacing}
  \begin{itemize}
    \item For controlling line spacing, I usually use the \texttt{setspace} package.
    \item Import it in the preamble: \mintinline{latex}{\usepackage{setspace}}
    \item Useful commands: \mintinline{latex}{\singlespacing}, \mintinline{latex}{\onehalfspacing}, \mintinline{latex}{\doublespacing}
    \item Useful environments: \texttt{singlespace}, \texttt{onehalfspace}, \texttt{doublespace}, \texttt{spacing}
  \end{itemize}
  \begin{minted}{latex}
\begin{spacing}{2.5}
  This paragraph has \\ huge gaps \\ between lines.
\end{spacing}
  \end{minted}
\end{frame}

\begin{frame}[fragile]{Quote-marks}
  In \LaTeX{}, quote-marks can go the wrong way if you're not careful!
  \begin{minted}{latex}
To `quote' in LaTeX
To ``quote'' in LaTeX
  \end{minted}
\end{frame}

\begin{frame}[fragile]{Paragraph Alignment}
  \begin{tabular}{|l|l|l|}
    \hline
    Alignment       & Environment         & Command                          \\ \hline
    Left justified  & \texttt{flushleft}  & \mintinline{latex}{\raggedright} \\
    Right justified & \texttt{flushright} & \mintinline{latex}{\raggedleft}  \\
    Center          & \texttt{center}     & \mintinline{latex}{\centering}   \\
    \hline
  \end{tabular}
\end{frame}

\begin{frame}[fragile]{Paragraph Indentation}
  \begin{itemize}
    \item By default, first paragraph after a heading is not indented, subsequent paragraphs are indented by \mintinline{latex}{\parindent}
    \item This follows typical Anglo-American publishing convention.
    \item To set this length, in preamble:
          \begin{minted}{latex}
\setlength{\parindent}{1cm} % Default 15pt
          \end{minted}
    \item You can use the \texttt{indentfirst} package to indent the beginning of every section
    \item To force indent a non-indented paragraph, use \mintinline{latex}{\indent} at the beginning of the paragraph.
    \item To force non-indent an indented paragraph, use \mintinline{latex}{\noindent}
  \end{itemize}
\end{frame}

\begin{frame}[fragile]{Adding paragraph skips}
  \begin{itemize}
    \item To make paragraphs boundary clear using zero indentation, vertical space between paragraphs is needed.
    \item Use the \texttt{parskip} package
  \end{itemize}
\end{frame}

\begin{frame}[fragile]{Verbatim Environment}
  Introduce text that will not be interpreted by the compiler in a monospaced font
  \begin{minted}{latex}
\begin{verbatim}
The verbatim environment
  simply reproduces every
 character you input,
including all  s p a c e s!
\end{verbatim}
  \end{minted}
\end{frame}

\begin{frame}[fragile]{Typesetting URLs}
  Use the \texttt{hyperref} package, with the \mintinline{latex}{\url{https://stonks.trade}} command

  If you want coloured hyperlink instead of box, set option \texttt{colorlinks} when using the \texttt{hyperref} package:
  \begin{minted}{latex}
\usepackage[colorlinks]{hyperref}
  \end{minted}
\end{frame}

\section{Mathematics}
\subsection{}
\begin{frame}{Mathematics}
  Knuth's motivation to develop \TeX{} among others was to allow simple construction of mathematical formulae that looks professional when printed.

  Typesetting Mathematics is one of \LaTeX{}'s greatest strengths
\end{frame}

\begin{frame}[fragile]{Getting started}
  I usually use the \texttt{mathtools} package to provide more powerful and flexible commands than plain \LaTeX{}

  \begin{minted}{latex}
\usepackage{mathtools}
  \end{minted}
\end{frame}

\begin{frame}[fragile]{Environments}
  \LaTeX{} provides displayed equation environment (\texttt{displaymath}), where the formulae are on a line by themselves.

  Short hand\footnote{DO NOT use \mintinline{latex}{$$...$$}, it is an older \TeX{} syntax that causes problems and is not officially supported by \LaTeX{}}: \mintinline{latex}{\[e^{i \pi} + 1 = 5\]}

  To get automatically numbered equations, use the \texttt{equation} environment:
  \begin{minted}{latex}
\begin{equation}
e^{i \pi} + 1 = 0
\end{equation}
  \end{minted}
\end{frame}

\begin{frame}[fragile]{Inline vs Displayed Equations}
  However, if you want to get an inline formula, use the \texttt{math} environment or the shorthand\footnote{There also exists the \LaTeX{} shorthand \mintinline{latex}{\(...\)}}:
  \begin{minted}{latex}
$e^{i \pi} + 1 = 0$
  \end{minted}

  These work on some flavours of Markdown too, e.g. \url{https://hackmd.io}
\end{frame}

\begin{frame}[fragile]{Maths Symbols}
  A pretty good list at \url{https://en.wikibooks.org/wiki/LaTeX/Mathematics\#List_of_mathematical_symbols}

  You can also use detexify: \url{http://detexify.kirelabs.org/}

  Or even cooler: \url{https://mathpix.com/}
\end{frame}


\begin{frame}{Powers and indices}
  Use the caret (\^{}) to raise something, and underscore (\_) to lower.

  If more than one expression is raised or lowered, group them using curly braces

  Exercise: typeset this

  $k_{n + 1} = n^2 + k_n^2 - k_{n - 1}$
\end{frame}

\begin{frame}[fragile]{Fractions and Binomials}
  \begin{minted}{latex}
$\frac{x^2}{y^3}$

$\binom{n}{r}$
  \end{minted}

  $\frac{x^2}{y^3}$

  $\binom{n}{r}$
\end{frame}

\begin{frame}[fragile]{Roots}
  \begin{minted}{latex}
$\sqrt[n]{1 + x + x^2 + x^3 + \dots + x^n}$
  \end{minted}

  $\sqrt[n]{1 + x + x^2 + x^3 + \dots + x^n}$
\end{frame}

\begin{frame}[fragile]{Sums and Integrals}
  Use the \mintinline{latex}{\sum} and \mintinline{latex}{\int} for sum and integral respectively, with the limits specified using caret and underscore.

  Use \mintinline{latex}{\limits} if you want the limits specified above and below the symbol in inline mode, or use displayed equation mode.

  \begin{columns}
    \begin{column}{0.8\textwidth}
      \begin{minted}{latex}
$\sum_{i=1}^{10} t_i$
      \end{minted}
    \end{column}
    \begin{column}{0.2\textwidth}
      $\sum_{i=1}^{10} t_i$
    \end{column}
  \end{columns}

  \begin{columns}
    \begin{column}{0.8\textwidth}
      \begin{minted}{latex}
$\sum\limits_{i=1}^{10} t_i$
      \end{minted}
    \end{column}
    \begin{column}{0.2\textwidth}
      $\sum\limits_{i=1}^{10} t_i$
    \end{column}
  \end{columns}

  Use \mintinline{latex}{\,} for a small space

  \begin{columns}
    \begin{column}{0.8\textwidth}
      \begin{minted}{latex}
$\int_0^\infty e^{-x}\,dx$
      \end{minted}
    \end{column}
    \begin{column}{0.2\textwidth}
      $\int_0^\infty e^{-x}\,dx$
    \end{column}
  \end{columns}

  \begin{columns}
    \begin{column}{0.8\textwidth}
      \begin{minted}{latex}
$\int\limits_0^\infty e^{-x}\,dx$
      \end{minted}
    \end{column}
    \begin{column}{0.2\textwidth}
      $\int\limits_0^\infty e^{-x}\,dx$
    \end{column}
  \end{columns}

\end{frame}

\begin{frame}[fragile]{Other big commands}
  Note that this also applies to other ``big'' commands like \mintinline{latex}{$\prod$} ($\prod$), \mintinline{latex}{$\bigcup$} ($\bigcup$), \mintinline{latex}{$\bigcap$} ($\bigcap$), etc.
\end{frame}

\begin{frame}[fragile]{Brackets, braces, delimiters}
  \begin{minted}{latex}
$( a ), [ b ], \{ c \}, | d |, \| e \|, \langle f \rangle, \lfloor g \rfloor, \lceil h \rceil, \ulcorner i \urcorner$
      \end{minted}
  $( a ), [ b ], \{ c \}, | d |, \| e \|, \langle f \rangle, \lfloor g \rfloor, \lceil h \rceil, \ulcorner i \urcorner$
\end{frame}

\begin{frame}[fragile]{Automatic sizing}
  \begin{minted}{latex}
  $P\left(A=2\middle|\frac{A^2}{B}>4\right)$

  $P(A=2|\frac{A^2}{B}>4)$
      \end{minted}
  $P\left(A=2\middle|\frac{A^2}{B}>4\right)$

  $P(A=2|\frac{A^2}{B}>4)$
\end{frame}

\begin{frame}{Exercises}
  $\binom{n}{r} = {}_nC_r = \frac{n!}{r!(n-r)!}$, ${}_nC_r \times r! = {}_nP_r$

  $\lim\limits_{n \rightarrow \infty}\left|\frac{a_{n+1}}{a_n}\right| = \rho$

  $\frac{d^2y}{dx^2}+p(x)\frac{dy}{dx}+q(x)y=F(x)$

  $\{x \mid x \in \mathbb{R}^+, -1 \leq x \leq 1\}$
\end{frame}

\begin{frame}{Resources}
  Wikibooks provide some good resources: \url{https://en.wikibooks.org/wiki/LaTeX}

  So does overleaf: \url{https://www.overleaf.com/learn/latex/Main_Page}
\end{frame}

\section{Conclusion}
\subsection{}
\begin{frame}
  \frametitle{Talk to us!}

  \begin{itemize}
    \item \textbf{Feedback form}: \url{https://bit.ly/2020ht6}
    \item Today is the last in the Hacker Tools series this semester, but do join our other events:
          \begin{itemize}
            \item Friday Hacks \#196: Resurgence of AI in the future of travel and tourism, and Stream Processing with Kafka Stream
            \item hackerschool: Advanced Git (Saturday)
          \end{itemize}
    \item Telegram: \url{https://t.me/nushackers} (@nushackers)
  \end{itemize}
\end{frame}

\end{document}
